%!TEX root = ../main.tex

	In the case of deterministic models, one of the problems taken into
account is to determine under what conditions the endemic equilibrium point is 
attractor or asymptotically stable. In the case of stochastic models, said 
endemic equilibrium point is not an equilibrium point. To determinate the 
persistence in the stochastic cases, we use the following definition.

	So how do we determine if the disease is going to persist? In this section 
we will give the conditions under which the difference between the solution of 
the system \eqref{system_3} and $ (S_p^*, L_p^*, I_p^*, S_v^*, I_v^*)^\top$ is 
small if the noise is weak, reflecting that the disease is prevalent.
\begin{definition}\label{def_per}
	Let $x = (S_p,L_p,I_p,S_v,I_v)^{\top}$ be the solution of 
	system \eqref{system_3}. We said that this solution process is persistent in mean if	
	\begin{equation}
	    \liminf_{
	    	t \to
	    	\infty
	    }
	    \frac{1}{t}
	    \int_0^t x(r) dr >0,
	    \qquad a.s.
    \end{equation}
\end{definition}
For establish the persistent of the endemic equilibrium point of the system 
\eqref{system_3}, we need consider the opposite conditions of 
Theorem *%\ref{theorem_2}. 
Our analysis require the following hypothesis.
\info{Who is theorem 2. Change label to a explicit meanin not fucking numbers}
\begin{enumerate}[(A)]
	\item
	According to Theorem %\ref{theorem_2}, 
	we need consider  
	$$
		\frac{\beta_p^2+r^2}{2 \sigma_p^2} + 
		2\beta_p - r>0,
	$$ 
	\item
	and
	$$
		\frac{\beta_v^2}{2 \sigma_v^2}+\beta_v-\gamma +\theta\mu>0
	$$.
\end{enumerate}
The following Theorem gives a upper bounds for the system \eqref{system_3}.
%
\info{Disucssuion about relation between $R_0^D$ $R_0^S$}
\begin{theorem}
	\label{Upper_Bound}
	Let $R^d_0>1$ and conditions (A)-(B) holds. 
	Consider the endemic deterministic fixed point 
	$(S_p^*,L_p^*,I_p^*,S_v^*,I_v^*)^\top$. Then 
	\begin{equation}\label{UP-1}
		\begin{aligned}
			&\limsup\limits_{t \rightarrow 	\infty}
			\frac{1}{t}
			\int_{0}^{t}( r
				\left(
					1 - 2 \rho_1
				\right)
				\left(
					(S_p - S_p^*) ^ 2 + 
					(L_p - L_p^*) ^ 2 + 
					(I_p - I_p^*) ^ 2
				\right)
			\\
			&+
				\gamma
				\left(
					1 - 2 \rho_2
				\right)
				(S_v - S_v) ^ 2 - 
				\gamma
				\left(
					1 - 
					\frac{1}{4 \rho_2}
				\right)
				(I_v - I_v)^2)ds
			\\
			&\leq
				K_2
				\alpha_1 + K_1
				\alpha_2 + 
				\frac{1}{2}
				\left(
					\sigma_p ^ 2
					( L_p ^ *K_1 + I_p ^ *K_2 + 2N_p^2)
					+3 N_v^2 \sigma_v^2
				\right)
				\qquad\mbox{a.s.}
		\end{aligned}
	\end{equation}
%
	where 
	$
		K_1 = \frac{N_p^2}{L_p^*}
	$, 
	$
		K_2 = \frac{N_p^2}{I_p^*}
	$, 
	$
		\rho_1 \in (0, \frac{1}{2})
	$ and 
	$
		\rho_2\in (\frac{1}{4}, \frac{1}{2})
	$.
\end{theorem}
%
\begin{proof}
	By hypothesis 
	$
		(S_p^*, L_p^*, I_p^*, S_v^*, I_v^*) ^ \top
	$ is the endemic equilibrium of system \eqref{system_1}, we have
	\begin{equation}
		\begin{aligned}
			rN_p
				&=
					rS_p^* + 
					\frac{\beta_p}{N_v}S_p^*I_v^*,
					\quad\quad 
					\frac{\beta_p}{N_v}S_p^*I_v^* = (b+r)L_p^*,
			\\
			bL_p^*
				&=
					rI_p^*, 
					\quad\quad 
					(1-\theta)
					\mu = \frac{\beta_v}{N_p}S_v^*I_p^*+\gamma S_v^*,
			\\
					\theta\mu &= 
					\gamma I_v^* -\frac{\beta_v}{N_p}S_v^*I_p^*.
		\end{aligned}
	\end{equation}
	Let consider the following Lyapunov function
	\begin{align*}
		V(S_p,L_p,&I_p,S_v,I_v) = 
				K_1
				\left(
					L_p - L_p^* - 
					L_p^*\log
					\left(
						\frac{L_p}{L_p^*}
					\right)
				\right) + 
				 K_2
				\left(
				 	I_p - I_p^* - I_p^ * 
				 	\log
					\left(
						\frac{I_p}{I_p^*}
				 	\right)
				\right)
			\\
				&+
				\frac{1}{2}
				\left(
					(S_p - S_p^*) + 
					(L_p - L_p^*) + 
					(I_p - I_p^*)
				\right)^2 + 
				\frac{1}{2}
				\left(
					(S_v - S_v^*) + 
					(I_v - I_v^*)
				\right)^2		
	\end{align*}
	We can rename the Lyapunov function as the follows
	\begin{equation}
		V(S_p,L_p,I_p,S_v,I_v)= K_1V_1+K_2V_2+V_3+V_4,
	\end{equation}
%
	and we work with each $V_i$. For $V_1$, we have
	\begin{align*}
		\mathcal{L} V_1 
			&= 
				\left(
					1 - 
					\frac{L_P^*}{L_p}
				\right)
				\left(
					\frac{\beta_p}{N_v}S_pI_v - 
					(b+r) L_p
				\right) + 
				\frac{1}{2}
				\frac{\sigma_p^2S_p^2L_p^*}{N_p^2}
			\\
			&=
				\left(
					1 - \frac{L_P^*}{L_p}
				\right)
				\left(
					\frac{\beta_p}{N_v} S_pI_v - 
					\frac{\beta_p}{N_v} S_p ^* I_v ^* 
					\frac{L_p}{L_p^*}
				\right) + 
				\frac{1}{2}
				\frac{\sigma_p^2S_p^2L_p^*}{N_p^2}
			\\
			&=
				\frac{\beta_p}{N_v}
				\left(
					1 - 
					\frac{L_P^*}{L_p}
				\right)
				\left(S_pI_v - 
					\frac{S_p^*I_v^*L_p}{L_p^*}
				\right) + 
				\frac{1}{2}
				\frac{\sigma_p^2S_p^2L_p^*}{N_p^2}
			\\
			&=
				\frac{\beta_p}{L_pN_v}
				\left(
					L_p - L_p^*
				\right)
				\left(
					S_pI_v - S_p ^ *I_v^*
					\frac{L_p}{L_p^*}
				\right) + 
				\frac{1}{2}
				\frac{\sigma_p^2 S_p^2 L_p^*}{N_p^2}.
	\end{align*}
%
Now, for $V_2$ we have
%
	\begin{align*}
		\mathcal{L}V_2 
			&= 
				\left(
					1 - 
					\frac{I_P^*}{I_p}
				\right)
				\left(
					bL_p - r I_p 
				\right) + 
				\frac{1}{2}
				\frac{\sigma_p^2 S_p^2 I_p^*}{N_p^2}
			\\
			&= 
				\frac{1}{I_p}
				(I_p-I_p^*)
				\left(
					\frac{rI_p^*}{L_p^*} - 
					rI_p
				\right) + 
				\frac{1}{2}
				\frac{\sigma_p^2 S_p^2 I_p^*}{N_p^2}
			\\
			&=
				-\frac{r}{I_p} (I_p - I_p ^ *)
				\left(I_p - \frac{I_p ^* }{L_p^*}
				\right) + 
				\frac{1}{2} 
				\frac{\sigma_p^2 S_p^2 I_p ^ *}{N_p^2}.
	\end{align*}
	%
	For $V_3$, we obtain
	\begin{align*}
		\mathcal{L}V_3
			&= 
				\left(
					(S_p - S_p ^* ) + 
					(L_p - L_p ^* ) +
					(I_p - I_p ^*)
				\right)
				\left(
					-\frac{\beta_p}{N_v}S_pI_v + 
					r N_p - rS_p 
				\right.\\
			&+
				\left.
					\frac{\beta_p}{N_v} 
					S_pI_v - (b+r)L_p + bL_p-rI_p
				\right) + 
				\sigma_p^2N_p^2
				\\
			= &
				\left(
					(S_p - S_p^*) + 
					(L_p - L_p^*) + 
					(I_p - I_p^*)
				\right)
				( r N_p - r S_p - r L_p - r I_p)
			\\
				& +
				\sigma_p^2N_p^2
			\\
			&=
				\left(
					(S_p - S_p ^*) + 
					(L_p - L_p ^*) + 
					(I_p - I_p^*)
				\right)
				( r I_p ^* + r L_p^* + rS_p^* - rS_p - r L_p-r I_p) + 
				\sigma_p^2 N_p^2
			\\
			&=
				\left(
					(S_p - S_p ^*) + 
					(L_p - L_p ^*) +
					(I_p - I_p ^*)
				\right)
				(
					-r (S_p - S_p ^*) -
					r (L_p - L_p^*) - 
					r(I_p-I_p^*)
				) + 
				\sigma_p^2 N_p^2
			\\
			&=
				-r
				\left(
					(S_p - S_p ^*) + 
					(L_p - L_p ^*) +
					(I_p - I_p^*)
				\right) ^ 2 + 
				\sigma_p^2 N_p^2.
	\end{align*}
	%
	For the last function $V_4$, we have
	%
	\begin{align*}
		\mathcal{L}V_4
			&= 
				\left(
					(S_v - S_v ^*) + 
					(I_v - I_v ^*)
				\right)
				\left( - 
					\frac{\beta_v}{N_p} S_vI_p -
					\gamma S_v + 
					(1 - \theta) 
					\mu 
				\right.\\
			&+
				\left.
					\frac{\beta_v}{N_p}
					S_v I_p - 
					\gamma I_v + 
					\theta \mu 
				\right) + 
				\frac{3}{2}
				\sigma_v ^ 2 N_v ^ 2
				\\
			&=
				\left(
					(S_v - S_v^*) + 
					(I_v - I_v^*)
				\right)
				(-\gamma S_v + \gamma S_v - \gamma I_v + \gamma I_v^*) + 
				\frac{3}{2}
				\sigma_v ^ 2 N_v ^ 2
			\\
			&=
				\left(
					(S_v - S_v ^*) + 
					(I_v - I_v ^*)
				\right)
				\left(
					-\gamma (S_v - S_v^*)
					-\gamma (I_v - I_v^*)
				\right) + 
				\frac{3}{2}
				\sigma_v ^ 2 N_v ^ 2
				\\
			&=
				-\gamma 
				\left(
					(S_v - S_v ^*) + 
					(I_v - I_v^*)
				\right) ^2 + 
				\frac{3}{2}
				\sigma_v ^2 N_v^2.
	\end{align*}
	%
	Then, we can bound the diffusion operator as follows
	%
	\begin{align*}
		\mathcal{L}V 
			&\leq 
				-r 
				\left(
					(S_p - S_p ^*) + 
					(L_p - L_p ^*) +
					(I_p - I_p ^*)
				\right) ^2 - 
				\gamma 
				\left(
					(S_v - S_v) + 
					(I_v - I_v)
				\right) ^ 2
			\\
			&
				-\frac{K_2 r}{I_p} (I_p - I_p ^*)
				\left(
					I_p - 
					\frac{I_p^*}{L_p^*}
				\right) + 
				\frac{\beta_p K_1}{N_vL_p}
				(L_p - L_p ^*)
				\left(
					S_pI_v - S_p ^* I_v^*
					\frac{L_p}{L_p^*}
				\right)
			\\
			&+
				\frac{1}{2}
				\left(
					\sigma_p^2
					(L_p ^* K_1 + I_p ^* K_2 + 2N_p ^ 2) + 
					3 N_v ^ 2 
					\sigma_v ^ 2
				\right)
	\end{align*}
%
	We need bound the term,
	$
		-\dfrac{K_2r}{I_p} (I_p-I_p^*)
		\left(
			I_p - \frac{I_p^*}{L_p^*}
		\right)
	$, 
	then
	\begin{align*}
		-\frac{K_2r}{I_p}
		(I_p - I_p ^*)
		\left(
			I_p - \frac{I_p^*}{L_p^*}
		\right)
			&=
				-\frac{K_2 r}{I_p}
				\left(
					I_p^2 - 
					\frac{I_pI_p^*}{L_p} - I_p^*I_p + 
					\frac{{I_p ^*}^2}{L_p^*}
				\right)
			\\
			&= 
				-K_2r
				\left(
					I_p - 
					\frac{I_p^*}{L_p^*} - 
					I_p ^* +
					\frac{{I_p^*}^2}{I_pL_p^*}
				\right)
			\\
			&\leq
				K_2 r 
				\left(
					\frac{I_p^*}{L_p^*} + I_p^*
				\right).
	\end{align*}
%
	Define $\alpha_1 := r\left(\frac{I_p^*}{L_p^*}+I_p^*\right)$, then
	\begin{align*}
		-\frac{K_2r}{I_p}
		(I_p - I_p ^*)
		\left(
			I_p - \frac{I_p ^*}{L_p ^*}
		\right)\leq K_2\alpha_1.
	\end{align*}
%
	Now the term 
	$
	\frac{\beta_p K_1}{N_vL_p}
	(L_p - L_p^*)
	\left(
		S_p I_v - S_p ^* I_v ^*
		\frac{L_p}{L_p^*}
	\right)
	$ can be bound as
	%
	\begin{align*}
		\frac{\beta_p K_1}{N_vL_p}
		(L_p - L_p^*)
		\left(
			S_pI_v - S_p ^* I_v ^* 
			\frac{L_p}{L_p^*}
		\right)
			&=
				\frac{\beta_p K_1}{N_v L_p}
					\left(
						L_p S_p I_v - S_p ^* I_v ^* 
						\frac{L_p ^2 }{L_p ^* } - L_p ^*S_pI_v + 
						S_p ^* I_v ^* L_p
					\right)
				\\
			&=		
				\frac{\beta_p K_1}{N_v}
				\left(S_p I_v - S_p ^* I_v ^* 
					\frac{L_p}{L_p ^* } - 
					\frac{L_p ^* }{L_p} S_pI_v + S_p^*I_v^*
				\right)
				\\
			&\leq
				\frac{\beta_p K_1}{N_v}
				\left(
					S_p I_v - S_p ^* I_v ^* 
				\right).	
	\end{align*}
%
	Since 
	$ 
		S_p,
		S_p ^* 
		\leq N_p
	$ 
	and 
	$ 
		I_v, I_v ^* \leq N_v
	$, this imply that
	\begin{align*}
		\frac{\beta_p K_1}{N_vL_p}
		(L_p - L_p ^* )
		\left(
			S_pI_v - S_p ^* I_v ^* 
			\frac{L_p}{L_p ^* }
		\right)
			&\leq
				2\frac{\beta_p K_1 N_p}{N_v}.
	\end{align*}
	%
	Define $\alpha_2:=2\frac{\beta_p N_p}{N_v}$, then
	%
	\begin{align*}
		\frac{\beta_p K_1}{N_vL_p}(L_p - L_p^*)
		\left(
			S_p I_v - S_p ^* I_v ^*
			\frac{L_p}{L_p^*}
		\right)
			&\leq
				K_1\alpha_2.
	\end{align*}
	%
	Therefore we can bound the diffusion operator $\mathcal{L} V$ as follows
	%
	\begin{align*}
		\mathcal{L}V 
			&\leq 
				-r 
				\left(
					(S_p - S_p ^*) + 
					(L_p - L_p ^*) +
					(I_p - I_p ^*)
				\right)^2 - 
				\gamma 
				\left(
					(S_v - S_v) + 
					(I_v - I_v)
				\right) ^ 2
				\\
			&+
				K_2 \alpha_1 + 
				K_1 \alpha_2 + 
				\frac{1}{2}
				\left(
					\sigma_p ^2 
					(
						L_p ^* K_1 + 
						I_p ^* K_2 + 
						2 N_p ^ 2
					) + 3 N_v ^ 2 
					\sigma_v ^ 2
				\right)
				\\
			&\leq 
				-3r 
				(S_p - S_p ^*) ^ 2 - 
				3r 
				(L_p - L_p ^*) ^2 - 
				3r
				(I_p - I_p ^*) ^ 2 - 
				2 \gamma 
				(S_v - S_v) ^ 2 - 
				2 \gamma 
				(I_v - I_v) ^ 2 
				\\
			&+
				K_2 \alpha_1 + 
				K_1 \alpha_2 + 
				\frac{1}{2}
				\left(
					\sigma_p ^ 2
					(
						L_p ^* K_1 + 
						I_p ^* K_2 + 
						2N_p ^ 2
					) 
					+ 3 N_v ^ 2
					\sigma_v ^ 2
				\right).
	\end{align*}
	By the Young's inequality we obtain that,
	\begin{align*}
	\mathcal{L} V 
		&\leq 
			-r
			\left(
				1 - 
				\frac{1}{2\rho_1} - 
				2 \rho_1 
			\right)
			\left(
				(S_p - S_p ^ *) ^ 2 + 
				(L_p - L_p ^ *) ^ 2 +
				(I_p - I_p ^ *) ^ 2
			\right)
			\\
			&-
				\gamma
				\left(
					1 - 2 
					\rho_2
				\right)
				(S_v - S_v) ^ 2 - 
				\gamma
				\left(
					1 - \frac{1}{4\rho_2}
				\right)
				(I_v - I_v) ^ 2
				\\
		&+
			K_2 \alpha_1 + 
			K_1 \alpha_2 + 
			\frac{1}{2}
			\left(
				\sigma_p ^ 2
				(
					L_p ^* K_1 + 
					I_p ^ *K_2 + 
					2N_p ^ 2
				) + 
				3N_v ^ 2 
				\sigma_v ^ 2
			\right)
		\\
		&\leq 
			-r
			\left(
				1 - 2 \rho_1 
			\right)
			\left(
				(S_p - S_p ^ *) ^ 2 + 
				(L_p - L_p ^ *) ^ 2 + 
				(I_p - I_p ^ *) ^ 2
			\right)
		\\
			&-
				\gamma
				\left( 
					1 - 2 \rho_2
				\right)
				(S_v - S_v) ^ 2 - 
				\gamma
				\left( 
					1 - \frac{1}{4\rho_2}\right)(I_v-I_v)^2\\
			&+
				K_2 \alpha_1 + K_1
				\alpha_2 + 
				\frac{1}{2}
				\left( 
					\sigma_p ^ 2
					(
						L_p ^* K_1 + 
						I_p ^* K_2 + 
						2 N_p ^ 2
					) + 
					3 N_v ^ 2
					\sigma_v ^ 2
					\right).
	\end{align*}
	Define $F(t)$ as
	\begin{align*}
		F(t)
			&:=
				-r 
				\left(
					1 - 2 \rho_1
				\right)
				\left(
					(S_p - S_p ^* ) ^ 2 + 
					(L_p - L_p ^* ) ^ 2 +
					(I_p - I_p ^*) ^ 2
				\right)
		\\
			&-
				\gamma
				\left(
					1 - 2 \rho_2
				\right)
				(S_v - S_v) ^ 2 - 
				\gamma
				\left(
					1 - 
					\frac{1}{4\rho_2}
				\right)
				(I_v - I_v) ^ 2
		\\
			&+
				K_2 \alpha_1 + K_1
				\alpha_2 + 
				\frac{1}{2}
				\left(
					\sigma_p ^ 2
					( 
						L_p ^* K_1 + 
						I_p ^* K_2 + 
						2N_p ^2
					) + 
					3N_v ^ 2
					\sigma_v ^ 2
				\right),
	\end{align*}
%
	therefore
	\begin{align*}
		dV
			&\leq
				F(t)dt + 
				\left(
					\frac{
						S_p
						\left( 
							\sigma_p L_p + 
							\sigma_p I_p 
						\right)
					}{N_p}
				\right)
				\left(
					 1 - 
					 	\frac {N_p}{S_p} - 
				\frac{\sigma_p S_p L_p}{N_p} - 
				\frac{\sigma_p S_p I_p}{N_p}
				\right)
				 dB_p(t)
				\\
			&-
				\frac{\sigma_v I_v \beta_p N_p dB_v(t)}{\gamma N_v}
	\end{align*}
	%
	Integrating both sides from 0 to $t$ yields
	\begin{align*}
	 	V_3(t) - 
	 	V_3(0)
	 	\leq
	 	&	
	 		\int_{0} ^ {t}
	 				F(s)ds +
	 		\\ 
	 		&
	 		\int_{0} ^ {t}
	 				\left(
	 					\frac{
	 						S_p 
	 						\left(
	 							\sigma_p L_p + 
	 							\sigma_p I_p 
	 						\right)
	 					}{N_p}
			 			\left(
			 				1 - 
			 				\frac{N_p}{S_p}
			 			\right) - 
			 				\frac{
			 					\sigma_p S_p L_p
			 				}{N_p}
			 			\right.
			% 	\\
	 		-
	 			\left.
	 				\frac{
	 					\sigma_p S_p I_p
	 				}{N_p}
	 			\right)  
	 			dB_p(s)
	 		\\
	 		&
	 			-
	 			\int_{0} ^ {t}
	 				\frac{\sigma_v I_v \beta_p N_p}{\gamma N_v} dB_v(s)
	 \end{align*}
	%
	Let 
	 \begin{align*}
	 	M_1(t)&:=
	 		\int_{0} ^ {t}
	 			\left( 
	 				\frac{
	 					S_p 
	 					\left(
	 						\sigma_p L_p + 
	 						\sigma_p I_p 
	 					\right)
	 				}{N_p}
	 				\left(
	 					1 - 
	 					\frac{N_p}{S_p}
	 				\right)
	 				- 
	 				\frac{
	 					\sigma_p S_p L_p
	 				}{N_p}
	 				\frac{\sigma_p S_p I_p}{N_p}
	 			\right) dB_p(s),
	 			\\
	 	M_2(t)&:=
	 		\int_{0} ^ {t}
	 			\frac{
	 				\sigma_v I_v 
	 				\beta_p N_p
	 				 dB_v(s)
	 			}{\gamma N_v}
	\end{align*}
%
 	and compute their quadratic variatiion, then
	\begin{align*}
	 	M_1(t)
	 	&:=
	 		\int_{0}^{t}
	 			\left( 			
	 		 	\frac{
	 		 		S_p
	 		 		\left(
	 		 			\sigma_p L_p + 
	 		 			\sigma_p I_p 
	 		 		\right)
	 		 	}{N_p}
	 		 	\left(
	 		 		1 - \frac{N_p}{S_p} 
	 		 	\right) -
	 		 	\frac{\sigma_p S_p L_p}{N_p}
	 		 	\frac{\sigma_p S_p I_p}{N_p}
	 			\right)
	 		 dB_p(s)
	 	\\
	 	&\leq
	 		\int_{0}^{t}
	 		\left( 
	 			\frac{
	 				S_p
	 				\left(
	 					\sigma_p L_p +
	 					\sigma_p I_p 
	 				\right)
	 			}{N_p}
	 			\left( 
	 				1 - 
	 				\frac{N_p}{S_p} 
	 			\right) 
	 		\right) dB_p(s)
	 	\\
	 	&\leq
	 		\int_{0}^{t}
	 		\left( 
	 			\frac{
	 				\sigma_p S_p
	 				\left(
	 					L_p + I_p
	 				\right)
	 			}{N_p}
	 			\left(
	 				\frac{S_p-N_p}{S_p} 
	 			\right) 
	 		\right) 
	 		dB_p(s)
	 	\\
	 	&\leq
	 		\int_{0}^{t}
	 		\left(
	 		 	-
	 		 	\frac{
	 		 		\sigma_p S_p
	 		 		\left(
	 		 			 L_p + I_p 
	 		 		\right)
	 		 	}{N_p}
	 		 	\left(
	 		 		\frac{L_p + I_p}{S_p}
	 		 	\right) 
	 		 \right) dB_p(s)
	 	\\
	 	&\leq
	 		\int_{0}^{t}
	 		4\sigma_p N_p dB_p(s).
	\end{align*}
	%%
	Similar for $M_2(t)$, we obtain
	 \begin{align*}
	 	M_2(t) &\leq
	 		\int_{0}^{t}\sigma_v\beta_pN_p dB_v(s),
	\end{align*}
	which are local continuous bounded martingale and $M_1(0)=M_2(0)=0$ with 
	quadratic variation finite. Then by Theorem 1.3.4 of [Mao's Book], we 
	obtain
	%
	 \begin{equation*}
	 	\lim
	 	\limits_{t \to \infty}
	 		\frac{M_1(t)}{t} = 0,
	 		\qquad \mbox{a.s., and}
	 \end{equation*}
	%
	\begin{equation*}
	 	\lim 
	 		\limits_{t\to \infty}
	 		\frac{M_2(t)}{t} = 0,
	 		\qquad \mbox{a.s.,}
	\end{equation*}
	%%
	by the $\liminf$ and $\limsup$ properties we have 
	\begin{align*}
	 	\liminf
	 	\limits_{t \to \infty} & 
	 	\frac{1}{t}
	 	\int_{0}^{t}
	 		F(s)ds 
	 	\geq 0 \qquad \mbox{a.s.}
	 	\\
	 	-\limsup_{t\to\infty}
	 	&
	 		\frac{1}{t}
	 		\int_{0} ^ {t}
	 			-F(s) ds \geq 
	 			0\qquad \mbox{a.s.},
	\end{align*}
	%%
	thus
	\begin{align*}
	 	\limsup_{t\to \infty}
	 	&
	 		\frac{1}{t}
	 		\int_{0} ^ {t} - 
	 		F(s)ds \leq 0
	 		\qquad \mbox{a.s.}
	 	\\
	\end{align*}
	%%
	%%
	Consequently,
	\unsure{check term like $(S_v - S_v)^2$}
	\begin{multline*}
	 	%&
	 	\limsup
	 	\limits_{t \to 	\infty}
	 	\frac{1}{t}
	 	\int_{0}^{t}
	 		(
	 			r
	 			\left(
	 				1 - 2 \rho_1
	 			\right)
	 			\left(
	 				(S_p - S_p ^*) ^ 2 + 
	 				(L_p - L_p ^*) ^ 2 +
	 				(I_p - I_p ^*) ^ 2
	 			\right)
	 	\\
	 	%&
	 	+
	 		\gamma
	 		\left(
	 			1 - 2 \rho_2
	 		\right)
	 		(S_v - S_v) ^ 2 - 
	 		\gamma
	 		\left( 1 - 
	 			\frac{1}{4\rho_2}
	 		\right)(I_v-I_v)^2)
	 		ds
	 	\\
	 	%&
	 	\leq
	 		K_2
	 		\alpha_1 + 
	 		K_1 \alpha_2 + 
	 		\frac{1}{2}
	 		\left(
	 			\sigma_p^2(
	 				L_p ^* K_1 + 
	 				I_p ^ *K_2 + 
	 				2N_p^2
	 			) + 
	 			3 N_v ^ 2
	 			\sigma_v ^ 2
	 		\right)
	 	\qquad 
	 	\mbox{a.s.}
 	\end{multline*}
 \end{proof}
\begin{remark}
 	The Theorem \ref{Upper_Bound} shows that, under some conditions, the 
 	distance between the solution 
 	$
 		X(t)=(S_p(t),L_p(t),I_p(t),S_v(t),I_v(t))^\top
 	$ 
 	and the fixed point 
 	$
 		X^*=(S_p^*,L_p^*,I_p^*,S_v^*,I_v^*)^\top
 	$ of system \eqref{system_1} has the following form:
 	\begin{equation*}
  		\limsup_{t\to\infty}
  		\frac{1}{t}
  		\int_{0} ^ {t}
  			\|X(s) - X ^*\| ^ 2 
  		ds
  		\leq 
  		C_1 + C_2
  		\|\sigma\| ^ 2,
  		\qquad a.s.,
 	\end{equation*}
	%%
 	where $C_1, C_2$ are positive constants. Although the solution ofsystem 
 	\eqref{system_3} does not have stability as the deterministic system, we 
 	obtain oscillations around deterministic fixed point
 	[*] provided 
 	$
 		C_1 + C_2 
 		\|\sigma\|^2
 	$ is sufficiently small. In this context, we 
 	consider the disease to persist.
\end{remark}
