%!TEX root = ../main.tex

In this section we will study when the disease can be extinguished, for this we will give the necessary conditions so that this phenomenon can occur through two different cases. The first case will be when due to the intensity of the noise.

The theorem presented below shows that under conditions on the parameters we can make the disease tend to become extinct.

\begin{theorem}\label{theorem_noise}[Extinction by noise]
	If
	\begin{align*}
		&\frac{\beta_p^2}{2\sigma_L^2}
		+\frac{r_2^2}{2\sigma_I^2}+2\beta_p-r_1<0,\\
		&\frac{\beta_v^2}{2\sigma_v^2}+\beta_v-\gamma+\theta\mu<0,
	\end{align*}		
	 then the disease will exponentially extinguish with probability one. that is, for any initial condition $(S_p(0),L_p(0),I_p(0),S_v(0),I_v(0))^\top\in \mathbb{R}_+^5$
	\begin{equation*}
		\limsup_{t\infty \rightarrow \infty}\frac{1}{t}\ln(L_p+I_p)< 0\,\, \mbox{and}\,\,\, \limsup_{t\infty \rightarrow \infty}\frac{1}{t}\ln(I_v)< 0\,\,\mbox{a.s.}
	\end{equation*}
\end{theorem}
\begin{proof}
	The main idea is apply the It\^{o} formula to a conveniently function and deduce conditions. Let $V(S_p,L_p,I_p)=\ln(L_p+I_p)$, then the It\^{o} formula gives
	
		\begin{align*}
			d \ln(L_p+I_p) 
				&=
					\left(\frac{1}{L_p+I_p}\right)\left(\frac{\beta_p}{N_v^\infty} S_p I_v-(b+r_1)L_p-\frac{1}{2}\sigma_L^2\frac{L_p^2}{(L_p+I_p)^2}\right)dt\\
				&-
					\sigma_L \frac{L_p}{L_p+I_p}dB_p(t)\\
				&\leq 
					\left(\frac{1}{L_p+I_p}\right)\left(\beta_p S_p-(b+r_1)-\frac{1}{2}\sigma_L^2\frac{L_p^2}{(L_p+I_p)^2}\right)dt\\
				&-
					\sigma_L \frac{L_p}{L_p+I_p}dB_p(t).\\
		\end{align*}
	Let $x:=\frac{L_p}{L_p+I_p}$, then
	
	\begin{align*}
		d \ln(L_p+I_p) 
			&\leq 
				\left(\beta_p\frac{S_p}{L_p+I_p}-(b+r_1)-\frac{1}{2}\sigma_L^2x^2\right)dt-\sigma_L xdB_p(t)\\
			&\leq
				\left(\beta_p\frac{N_p}{L_p+I_p}-(b+r_1)-\frac{1}{2}\sigma_L^2x^2\right)dt-\sigma_L xdB_p(t)\\
			&\leq
				\left(\beta_px+2\beta_p-(b+r_1)-\frac{1}{2}\sigma_L^2x^2\right)dt-\sigma_L xdB_p(t)\\
			&=
				\left(-\frac{1}{2}\sigma_L^2x^2+\beta_px+2\beta_p-(b+r_1)\right)dt-\sigma_L xdB_p(t).			
	\end{align*}
	
	Hence,
	
	\begin{align*}
		\ln(L_p+I_p)
			&\leq
				-\frac{\sigma_L^2}{2}\int_{0}^{t}\left(\left(x-\frac{\beta_p}{\sigma_L^2}\right)^2 +\frac{\beta_p^2}{2\sigma_L^2}+2\beta_p-(b+r_1)\right)du\\
			&-
				\int_{0}^{t}\sigma_L xdB_p(u)+\ln(L_p(0)+I_p(0)),
	\end{align*}
	
	which implies,
	
		\begin{align}\label{eq4.1}
			\frac{1}{t}\ln(L_p+I_p) 
				&\leq
					-\frac{\sigma_L^2}{2t}\int_{0}^{t}\left(x-\frac{\beta_p}{\sigma_L^2}\right)^2du+
					\frac{\beta_p^2}{2\sigma_L^2}-(b+r_1)+2\beta_p\nonumber\\
				&-
					\frac{1}{t}\int_{0}^{t}\sigma_L xdB_p(u)+\frac{1}{t}\ln(S_p(0)+L_p(0)+I_p(0)),
		\end{align}
	
	let $M_t :=\frac{1}{t}\int_{0}^{t}\sigma_L xdB_p(t)+\frac{1}{t}\ln(L_p(0)+I_p(0))$. Since the integral in the term $M_t$ is a martingale, the strong law of large numbers for martingales Mao, implies that 
	
	\begin{equation*}
		\lim\limits_{t \rightarrow \infty}M_t = 0\,\,\mbox{a.s.}
	\end{equation*}
	
	Thus, from relation (\ref{eq4.1}) we obtain
	
	\begin{align}\label{eq4.2}
		\limsup_{t\infty \rightarrow \infty}\frac{1}{t}\ln(L_p+I_p)<\frac{\beta_p^2}{2\sigma_L^2}+	2\beta_p-(b+r_1)
	\end{align}
	
	A similar argument also shows that
	
	\begin{align}\label{eq4.3}
		\limsup_{t\infty \rightarrow \infty}\frac{1}{t}\ln(L_p+I_p)<\frac{r_2^2}{2\sigma_I^2}+b
	\end{align}
	
	Through the equations (\ref{eq4.2}) and (\ref{eq4.3}), we obtain
	
	\begin{align*}
		\limsup_{t\infty \rightarrow \infty}\frac{1}{t}\ln(L_p+I_p)<\frac{\beta_p^2}{2\sigma_L^2}+\frac{r_2^2}{2\sigma_I^2}+	2\beta_p-r_1
	\end{align*}
	
	and 
	
	\begin{align*}
		\limsup_{t\infty \rightarrow \infty}\frac{1}{t}\ln(I_v)<\frac{\beta_v^2}{2\sigma_v^2}+\beta_v-\gamma+\theta\mu
	\end{align*}
\end{proof}

\begin{remark}
	Theorem \ref{theorem_noise} shows that, under certain conditions on the parameters can cause disease exponentially towards zero whenever the noise intensity is large enough.
\end{remark}

The next case of extinction of the disease is through the basic reproductive number. For the deterministic case, defining the basic reproductive number is done using the next generation matrix [Van der drish], but in the stochastic case it is not possible to give such a definition.

To define the stochastic reproductive number we will use the techniques used in [Agwar], in which, by means of algebraic procedures, this parameter can be defined. As our deterministic base structure this paramenters summarizes the behavior of extinction and persistence according to a threshold.

Our analysis needs the following function and conditions.
\begin{enumerate}[(H-1)]
	\item
	According to SDE \eqref{system_3}, replatin rates satisfies 
	$r=r_1+r_2$. 
	\item
	The replanting noise intesities are equal
	$\sigma_L = \sigma_I = \sigma_p$.
\end{enumerate}
Given a function $V\in C^{2,1}(\R^n\times\R_{+};\R)$, define an operator $LV:\R^n\times\R_{+}\rightarrow\R$ by

\begin{equation}\label{InfiOpera}
	\mathcal{L}[V(x,t)] = V_t(x,t)+V_x(x,t)f(x,t)+\frac{1}{2}trace(g^T(x,t)V_{xx}(x,t)g(x,t))
\end{equation}
%
which is called the diffusion operator of the It\^{o} process associated with the $C^{2,1}$ function $V$. With this diffusion operator, the It\^{o} formula can be written as

\begin{equation}\label{Itoformula}
dV(x(t),t) = \mathcal{L}V(x(t),t)dt+V_x(x(t),t)g(x(t),t)dB(t)\qquad a.s.
\end{equation}

We define the repoductive nomber of our stochastic model in SDE (\ref{system_3}) by

\begin{equation}\label{eq5}
	\mathcal{R}_0^s=\frac{\beta_p\beta_v}{\gamma r}
\end{equation}

\begin{theorem}\label{theorem_2}
	Let $(S_p(t),L_p(t),I_p(t),I_v(t))$ be the solution of SDE (\ref{system_3}) with initial values $(S_p(0),L_p(0),I_p(0),I_v(0))\in (0,N_p)\times(0,N_p)\times(0,N_p)\\\times(0,N_v)$. If $0\leq \mathcal{R}^s_0<1$, then the following conditions holds

	\begin{align*}
		\lim\limits_{t\rightarrow \infty}\frac{1}{t}\mathbb{E}\int_{0}^{t}\left[{r[\mathcal{R}^s_0-1]I_p-rS_p\left(1-\frac{S^0_p}{S_p}\right)^2-rL_p-\frac{\beta_p\beta_v}{\gamma}I_vI_p}\right]dr\leq \frac{1}{2}\sigma^2N_p,\, a.s.,
	\end{align*}
	namely, the infected individual tends to zero exponentially a.s, i.e the disease will die out with probability one.
\end{theorem}

\begin{proof}
	The proof consitst verify the hypotheses of Khasminskii Theorem [*] for the Lyapunov function
	
	\begin{align*}
		V(S_p,L_p,I_p,I_v) &= 
			\left(S_p-S_p^0-S_p^0\ln\frac{S_p}{S_p^0}\right)+L_p+I_p+\frac{\beta_p N_p}{\gamma N^\infty_v}I_v,
	\end{align*}
	
	Let $f$, $g$ respectively be the dirft and difussion of SDE (\ref{Positive3}).	Applyng the diffusion operator $\mathcal{L}$ we have
	
	\begin{align*}
		V_x f &=
				\left(1-\frac{S_p^0}{S_p}\right)\left(-\frac{\beta_p}{N^\infty_v}S_pI_v+ rN_p-r S_p\right)+\frac{\beta_p}{N^\infty_v}S_pI_v-(b+r)L_p\\
			  &+
				bL_p-rI_p+\frac{\beta_p N_p}{\gamma N^\infty_v}\left(\frac{\beta_v N_v}{N_p}I_p-\frac{\beta_v}{N_v^\infty}I_vI_p-\gamma I_v\right)\\
			  &=
			  	-rS_p\left(1-\frac{S_p^0}{S_p}\right)^2-\frac{\beta_p}{N_v^\infty}S_pI_v+\frac{\beta_p}{N_v^\infty}I_vS_p^0+\frac{\beta_p}{N_v^\infty}S_pI_v-r(L_p+I_p)\\
			  &+
			  	\frac{\beta_p N_p}{\gamma N^\infty_v}\left(\frac{\beta_v N_v}{N_p}I_p-\frac{\beta_v}{N_v^\infty}I_vI_p-\gamma I_v\right)\\
			  &=
			  	-rS_p\left(1-\frac{S_p^0}{S_p}\right)^2+\frac{\beta_p}{N_v^\infty}I_vS_p^0-r(L_p+I_p)+\frac{\beta_p N_p}{\gamma N^\infty_v}\frac{\beta_v N_v}{N_p}I_p\\
			  &-	
			  	\frac{\beta_p N_p}{\gamma N^\infty_v}\frac{\beta_v}{N_v^\infty}I_vI_p-\frac{\beta_p N_p}{\gamma N^\infty_v}\gamma I_v\\
	\end{align*}
	
	Then,
	
	\begin{align*}
		V_x f &=
				-rS_p\left(1-\frac{S_p^0}{S_p}\right)^2-\left[\frac{\beta_p N_p}{\gamma N^\infty_v}\gamma-\frac{\beta_p N_p}{N_v^\infty}\right]I_v+\left[\frac{\beta_p N_p}{\gamma N^\infty_v}\beta_v\frac{N_v^\infty}{N_p}-r\right]I_p\\
		 	  &-
				rL_p-\frac{\beta_p N_p}{\gamma N^\infty_v}\frac{\beta_v}{N_p}I_vI_p\\
			  &=
			  	-rS_p\left(1-\frac{S_p^0}{S_p}\right)^2 + \left[\frac{\beta_p N_p}{\gamma N^\infty_v}\beta_v\frac{N_v^\infty}{N_p}-r\right]I_p-rL_p-\frac{\beta_p N_p}{\gamma N^\infty_v}\frac{\beta_v}{N_p}I_vI_p\\
			  &=
			  	-rS_p\left(1-\frac{S_p^0}{S_p}\right)^2 + r\left[\frac{\beta_p\beta_v}{\gamma r}-1\right]I_p-rL_p-\frac{\beta_p\beta_v}{\gamma N^\infty_v}I_vI_p.
	\end{align*}
	
	Expressing the right hand side of above equation in term of the basic reproductive number, $\mathcal{R}^s_0$ we get
	
	\begin{align*}
		V_x f &=
		-rS_p\left(1-\frac{S_p^0}{S_p}\right)^2 + r\left[\mathcal{R}^s_0-1\right]I_p-rL_p-\frac{\beta_p\beta_v}{\gamma N^\infty_v}I_vI_p.
	\end{align*}

	Moreover,
	
	\begin{align*}
		\frac{1}{2}trace(g^TV_{xx}g) 
			&=
				\frac{1}{2} \sigma^2N_p\left(\frac{N_p-S_p}{S_p}\right)^2\\
			&\leq
				\frac{1}{2} \sigma^2 N_p.
	\end{align*}
	The stochastic terms are not necessary, because they do a martingale process and therefore, when we use integral and expectation they vanishing.
	
	Incorporation all terms calculate above, we obtain
	
		
	\begin{align*}
		dV(X) 
			&=
				-rS_p\left(1-\frac{S_p^0}{S_p}\right)^2 + r\left[\mathcal{R}^s_0-1\right]I_p-rL_p-\frac{\beta_p\beta_v}{\gamma N^\infty_v}I_vI_p+\frac{1}{2} \sigma^2N_p\left(\frac{N_p-S_p}{S_p}\right)^2\\
			&\leq
				-rS_p\left(1-\frac{S_p^0}{S_p}\right)^2 + r\left[\mathcal{R}^s_0-1\right]I_p-rL_p-\frac{\beta_p\beta_v}{\gamma N^\infty_v}I_vI_p+\frac{1}{2} \sigma^2 N_p.
	\end{align*}
	Define $\mathcal{L}V(X)$ as
	\begin{align*}
		\mathcal{L}V(X) 
			&=
				-rS_p\left(1-\frac{S_p^0}{S_p}\right)^2 + r\left[\mathcal{R}^s_0-1\right]I_p-rL_p-\frac{\beta_p\beta_v}{\gamma N^\infty_v}I_vI_p+\frac{1}{2} \sigma^2 N_p.
	\end{align*}
	
	Using It\^{o}'s formula and integrating $dV$ from $0$ to $t$ as well as taking expectation yield the following
	
	\begin{align*}
		0 
			&\leq
				\mathbb{E}V(t)-\mathbb{E}V(0)\leq\mathbb{E}\int_{0}^{t}LV(X(s))ds\\
			&\leq
				-\mathbb{E}\int_{0}^{t}\left[rS_p\left(1-\frac{S_p^0}{S_p}\right)^2 - r\left[\mathcal{R}^s_0-1\right]I_p+rL_p+\frac{\beta_p\beta_v}{\gamma N^\infty_v}I_vI_p\right]ds+ \frac{1}{2}\sigma^2 N_p
	\end{align*}
	
	Therefore,
	
	\begin{align*}
	\lim\limits_{t\rightarrow \infty}\frac{1}{t} 
	\mathbb{E}\int_{0}^{t}\left[-rS_p\left(1-\frac{S_p^0}{S_p}\right)^2 + r\left[\mathcal{R}^s_0-1\right]I_p-rL_p-\frac{\beta_p\beta_v}{\gamma N^\infty_v}I_vI_p\right]ds\leq \frac{1}{2}\sigma^2 N_p.
	\end{align*}
		
\end{proof}
\begin{remark}
	Theorem \ref{theorem_2} shows that, if the basic stochastic reproductive number $\mathcal{R}^s_0$ is less than one, we have the solutions $X(t)=(S_p (t), L_p (t), (t ) I_p (t), S_v (t), I_v (t))^{\top}$ tend to the equilibrium point $(N_p, 0,0, N_v^{\infty}, 0)^{\top}$, when $t\rightarrow \infty$.
\end{remark}