%!TEX root = ../main.tex
Thereom *.* of [Mao Book] assures ths existence of unique solution of 
\eqref{system_3} in a compact interval. Since we study asymptotic behaviour, 
we have to assure the existence of unique-globally-positive invariant solution 
of SDE (*). To this end, let $\R^n_+$ the first octant of $\R ^ n$ and 
consider  
$$	{
	\mathbf{E}:= 
		\left \{ 
			(S_p, L_p, I_p, S_v, I_v)^{\top} \in \R^5_+: \quad
			0\leq S_p + L_p + I_p \leq N_p, \quad
			S_v + I_v \leq \frac{\mu}{\gamma}
		\right \},
	}
$$
the following result prove that this set is positive invariant.
%
\begin{theorem}\label{existence-unique}
	For any initial values 
	$
		(S_p(0), L_p(0), I_p(0), S_v(0), I_v(0))^{\top}
		\in \mathbf{E}
	$, 
	exists unique a.s. invariant global positive solution to SDE 
	\eqref{system_3} in $\mathbf{E}$, that is,
	\begin{equation*}
		\probX{
			(L_p(t), I_p(t), S_v(t), I_v(t)) 
			\in 
			\mathbf{E}, \quad
			\forall t \geq 0
		} = 1.
	\end{equation*}
\end{theorem}
%
\begin{proof}
		Since the right hand side of system \eqref{system_3} are quadratic, 
	linear and constans terms, this imply that they are locally Lipschitz. We 
	know by [ref Mao], that for any initial condition
	$
		(
			S_p(0),
			L_p(0),
			I_p(0),
			S_v (0),
			I_v(0)
		)^{\top} \in \mathbf{E}
	$ 
	there is a unique maximal local solution 
	$(S_p(t), L_p(t), I_p(t), S_v(t), I_v(t))^{\top}$ 
	at $t\in [0,\tau_e)$, where 
	$\tau_e$ is the explosion time. Let $k_0>0$ be sufficiently large, and 
	define the stopping time
	\begin{multline}
		\label{eqn:invariatn_set}
		\tau_k = %
			\inf
			\left\{
				t \in [0,\tau_e)
				: 
				L_p(t) \notin 
					\left(
						\frac{1}{k_0},
					 N_p - \frac{1}{k_0}
					\right)		
			%\right.
			%\\
			%\left.
			 \bigcup 
				I_p(t)
				\notin
				\left(
					\frac{1}{k_0}, 
					N_p-\frac{1}{k_0}
				\right)
			\right.
			\\
			\bigcup
			\left.		
					I_v(t) 
					\notin
					\left(
						\frac{1}{k_0},
						N_v - \frac{1}{k_0}
					\right)
			\right\},
		\end{multline}
%	
	\improvement{Give an argument}
	We know that $\tau_k \nearrow \tau_\infty$. 
	In other words, $\tau_\infty = \infty$ a.s. 
	implies
	\begin{equation}
		\label{eqn:invariance_prop}
		(
			S_p(t),
			L_p(t),
			I_p(t),
			S_v(t),
			I_v(t)
		)^{\top}\in \mathbf{E}
	\end{equation}
	 a.s. for all $t\geq 0$. Thus, 
	we  show that $\tau_\infty=\infty$ a.s. To this end,  we proced
	by contradiction. Suppose that the above statement is false for a given 
	time $t$, then there is 
	a pair of constants $T>0$ and $\epsilon  \in (0,1)$  such that some 
	component from $L_p,I_p,I_v$, or $L_p$, get-outs from its corresponding 
	interval
	$$
	\left(
		\frac{1}{k_0}, 
	N_{\bullet} - \frac{1}{k_0}
	\right), 
	$$
	that is, %
	$
		\P[\tau_\infty\leq T]>\epsilon 
	$. 
	Hence, there is an integer $k_1\geq k_0$ such that
%	
	\begin{equation}\label{Positive1}
		\P[\tau_k\leq T]>\epsilon,
		\quad \forall k\geq k_1.
	\end{equation}
	
		Define a function $V_p:(0,N_p)\rightarrow \mathbb{R}_+$ by
%	
	\begin{equation*}
		V_p(x) := 
			\frac{1}{x} + 
			\frac{1}{N_p-x}.
	\end{equation*}
%	
	According to the inifinitesimal operation $\mathcal{L}$
	see \autoref{app:} APPENDIX
	\improvement{Write auxiliar results in a fucking appendix}
	By diffusion operator, we have, for any $t\in [0,T]$ and $k\geq k_1$ 
%
	\begin{align*}
		\mathcal{L}[V_p(L_p)] 
			=&	
				\left[
					-\frac{1}{L_p^2} + 
					\frac{1}{(N_p - L_p) ^ 2}
				\right]
				\left[
					\beta_p S_p 
					\frac{I_v}{N_v} - 
					(b + r_1) L_p\right]
				\\
			&+
				\frac{1}{2}
				\left[
					\frac{2}{L_p^3} + 
					\frac{2}{(N_p - L_p)^3}
				\right]
				\sigma_p ^ 2
				\frac{L_p^2 S_p^2}{N_p^2}.
			\\
	\end{align*}
%	
	Expanding each term, we have
	\begin{align*}
		\mathcal{L}[V_p(L_p)] 
			&=	
				-\beta_p \frac{S_p I_v}{L_p^2 N_v} + 
				\beta_p \frac{S_p I_v}{(N_p -L_p) ^ 2 N_v} + 
				\frac{(b + r_1)}{L_p} - 
				\frac{(b+r_1)L_p}{(N_p-L_p)^2}
				\\
			&+
				\left[
					\frac{1}{L_p^3} + 
					\frac{1}{(N_p-L_p)^3}
				\right]
				\sigma_p ^ 2
				\frac{L_p^2 S_p^2}{N_p^2}.
				\\
	\end{align*}
%	
	Droping negative terms, we bound the above relation by
	\begin{align*}
		\mathcal{L}[V_p(L_p)] 
			&\leq	
				\beta_p 
				\frac{S_p}{(N_p - L_p) ^ 2} + 
				\frac{(b + r_1)}{L_p} + 
				\left[
					\frac{1}{L_p^3} + 
					\frac{1}{(N_p-L_p)^3}
				\right]
				\sigma_p ^ 2
				\frac{L_p ^ 2 S_p ^ 2}{N_p ^ 2}.
	\end{align*}
	Moreover we see that $S_p\leq N_p-L_p=S_p+I_p$, thus
	\info{Rewview this steap}
	\begin{align*}
		\mathcal{L}[V_p(L_p)] 
			&\le q		
				\frac{\beta_p}{N_p -L_p} + 
				\frac{(b + r_1)}{L_p} + 
				\sigma_p^2
				\left[
					\frac{1}{L_p} + 
					\frac{L_p^2}{N_p^2(N_p-L_p)}
				\right].
	\end{align*}
	And this implies that
	\info{explain why}
	\begin{align*}
		\mathcal{L}[V_p(L_p)] 
			&\leq		
			\frac{b+r_1}{L_p} + 
			\frac{\beta_p}{N_p -L_p} +  
			\sigma_p^2
			\left[
				\frac{1}{L_p} + 
				\frac{1}{N_p-L_p}
			\right].
	\end{align*}
	%
	Now define $C := (b+r_1) \vee \beta_p +\sigma_p^2$, we obtain the 
	following inequality
	\begin{align}\label{Positive2}
		\mathcal{L}[V(L_p)] 
			&\leq		
				C V_p(L_p).
	\end{align}
%
	
		By It\^{o}'s formula and applying expectation, we have, for any 
	$t \in [0,T]$ and $k\geq k_1$
	\begin{align*}
		\E V(L_p(t\wedge\tau_k)) &= 
			V(L_p(0)) + 
			\E 
				\int_{0}^{t\wedge\tau_k} \mathcal{L}[V(L_p(s))]
			ds.
	\end{align*}
	%
	By equation \eqref{Positive2} and Fubini's Theorem, we have
	\begin{align*}
		\E V(L_p(t\wedge\tau_k)) 
			&\leq 
				V(L_p(0)) + 
				C
				\int_{0}^{t}
					\E V(L_p(s\wedge\tau_k))
				ds.
	\end{align*}
	Applying the Gronwall inequality yields that	
	\begin{align}\label{Positive3}
		\E V(L_p(t\wedge\tau_k))&\leq V(L_p(0))e^{CT}.
	\end{align}

		Set 
	$
		\Omega_k = \{\omega : \tau_k\leq T\}
	$ for $k\geq k_1$, note that by relation in 
	\autoref{Positive1}, 
	$
		\P(\Omega_k) >  \epsilon
	$. For every 
	$
		\omega \in \Omega_k
	$, we have 
	$
		L_p(t,\omega) \in 
		\left(
			\frac{1}{k_0}, N_p - 
			\frac{1}{k_0}
		\right) ^ {\complement}
	$, and hence
	\begin{align*}
		V_p(L_p(t,\omega))
			&=
				\frac{1}{L_p} + 
				\frac{1}{N_p-L_p}
			\\
			&\geq 
				k + 
				\frac{1}{
					N_p - \frac{1}{k}}
			\\
			& \geq k.
	\end{align*}
%	
	It follows from equation \eqref{Positive3}, that
	\begin{equation*}
		V_p(L_p(0)) e^{CT}
			\geq 
			\E 
			\left[
				\1{\Omega_k} (\omega)
				V_p(L_p(\tau_k,\omega))
			\right]
			\geq k
			\prob (\Omega_k)\geq \epsilon k.
	\end{equation*}
%
	Thus, letting $k\rightarrow \infty$ leads to the contradiction
	\begin{equation*}
		\infty>V_p(L_p(0))e^{CT}\geq \infty.	
	\end{equation*}
%
	Therfore we  have $\tau_\infty=\infty$ a.s., and the proof is 
	complete. \qed
\end{proof}
