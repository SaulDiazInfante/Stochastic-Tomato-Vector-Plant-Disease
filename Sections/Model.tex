\section{Tomato Model}
En esta secci\'on, vamos a definir el modelo b\'asico que trabajaremos, consideraremos que las plantas se dividen en tres tipos: plantas susceptibles, latentes e infectadas. Las moscas blancas, las cuales llamaremos vectores, se dividen en susceptibles e infectadas.

Las plantas susceptibles pasan a ser plantas latentes cuando un vector infectado se alimenta de ella a una tasa de $\beta_p$, continuando el proceso cuando las plantas latentes se convierten en plantas infectadas a una tasa de $b$, en cada uno de estos casos consideraremos que estaremos revisando los cultivos para el cual removeremos plantas latentes e infectadas si se detecta que dicha planta esta infectada a una tasa de $r_1$ y $r_2$ respectivamente.

Plants become latent by
infected vectors,
replanting latent and
infected plants,
latent plants become
infectious plants,
vectors become infected by
infected plants,
vectors die or depart per
day,
immigration from
alternative hosts.

\begin{align}\label{system_1}
	\dot{S_p} &=
		-\beta_p S_p \frac{I_v}{N_v} + \tilde{r_1} L_p + \tilde{r_2} I_p \\
	\dot{L_p} &=
		\beta_p S_p \frac{I_v}{N_v} - b L_p - \tilde{r_1} L_p \\
	\dot{I_p} &=
		b L_p - \tilde{r_2} I_p \\
	\dot{S_v} &=
		-\beta_v S_v \frac{I_p}{N_p} -\tilde{\gamma} S_v +(1-\theta) \mu \\
	\dot{I_v} &=
		\beta_v S_v \frac{I_p}{N_p} -\tilde{\gamma} I_v + \theta \mu
\end{align}

donde $\beta_p$: tasa de infecci\'on de las plantas susceptibles mediante un vector infectado. $r_1$: tasa de replanteo de plantas latentes. $r_2$: tasa de replanteo de plantas infecciosas. $b$: tasa de latencia (planta latente se convierte en infecciosa). $\beta_v$: tasa de infecci\'on de los vectores susceptibles mediante una planta infectada. $\gamma$: tasa de muerte o alejamiento de los vectores, $\mu$: migraci\'on de los vectores de plantas hospederas alternas, $\theta$: proporci\'on de migraci\'on de los vectores.

\begin{theorem}\label{theorem_1}
	With the same notation of SDE (\ref{system_1}), let
	\begin{equation*}
	N_v(t) := S_v(t) + I_v(t),\, N_v^0 := S_v(0) + I_v(0),\, N_v^{\infty} := \frac{\mu}{\gamma}.
	\end{equation*}
	Then for any initial condition $N_v^0$ in $(0, N^\infty_v]$, the whole vector population satisfies
	\begin{equation*}
	N_v(t) = N^\infty_v +(N^0_v-N^\infty_v)e^{-\gamma t},\, t\geq 0.
	\end{equation*}
\end{theorem}

We first going to adimensionality the system (\ref{system_1}), with the following variable change:

\begin{equation*}
	x=\frac{S_p}{N_p}, y=\frac{L_p}{N_p}, z=\frac{I_p}{N_p}, v=\frac{I_p}{N_v}, w=\frac{I_v}{N_v}
\end{equation*}

Then, the system (\ref{system_1}) becomes

\begin{align}\label{system_2}
\dot{x} &=
-\beta_p x w + \tilde{r_1} y + \tilde{r_2} z \\
\dot{y} &=
\beta_p x w - (b + \tilde{r_1}) y \\
\dot{z} &=
b y - \tilde{r_2} z \\
\dot{v} &=
-\beta_v v z  +(1-\theta-v)\frac{\mu}{N_v} \\
\dot{w} &=
\beta_v v z + (\theta-w) \frac{\mu}{N_v}
\end{align}


Following [referencia], we are interested in a model where the replanting rate of plants, and died rate of vector, $r_1$, $r_2$ and $\gamma$ are now a random variables. This could be doubt to some stochastic environmental factor acts simultaneously on each plant in the crop. More precisely each replanting, died rate, makes

\begin{equation}\label{eq1}
	\tilde{r}_1 dt = r_1 dt+\sigma_LdB(t),
\end{equation}

\begin{equation}\label{eq2}
\tilde{r}_2 dt = r_2 dt+\sigma_IdB(t),
\end{equation}

\begin{equation}\label{eq3}
\tilde{\gamma} dt = \gamma dt+\sigma_vdB(t),
\end{equation}

potentially replanting, and vector death  in $[t, t + dt)$. Here $dB(t) =B(t+dt)-B(t)$is the increment of a standard Wiener process or Brownian motion. Note that $\tilde{r}_1$,$\tilde{r}_2$,$\tilde{\gamma}$ are just a random perturbations of $r_1$,$r_2$,$\gamma$, with $\mathbb{E}(\tilde{r}_1dt) = r_1dt$ and $Var(\tilde{r}_1dt) = \sigma_L^2dt$, $\mathbb{E}(\tilde{r}_2dt) = r_2dt$ and $Var(\tilde{r}_2dt) = \sigma_I^2dt$ and $\mathbb{E}(\tilde{\gamma}dt) = \gamma dt$ and $Var(\tilde{\gamma}dt) = \sigma_v^2dt$.

Thus, the stochastic tomato model is given by the following system of coupled Ito's SDE's

\begin{align}\label{system_3}
	d S_p &=
		\left(-\beta_p S_p \frac{I_v}{N_v} + r_1 L_p + r_2 I_p\right)dt + (\sigma_L L_p + \sigma_I I_p)dB(t) \\
	dL_p &=
		\left(\beta_p S_p \frac{I_v}{N_v} - b L_p - r_1 L_p\right)dt - \sigma_L L_p dB(t) \\
	d I_p &=
		\left(b L_p - r_2 I_p\right)dt -\sigma_I I_p dB(t) \\
	dS_v &=
		\left(-\beta_v S_v \frac{I_p}{N_p} -\gamma S_v +(1-\theta) \mu\right)dt - \sigma_v S_v dB(t) \\
	d I_v &=
		\left(\beta_v S_v \frac{I_p}{N_p} -\gamma I_v + \theta \mu\right)dt - \sigma_v I_v dB(t)
\end{align}

and the corresponding change of variable, the system (\ref{system_3}) can be reewriting as

\begin{align}\label{system_4}
	d x(t) &=
		(-\beta_p x w + r_1 y + r_2 z)dt +(\sigma_L y + \sigma_I z)dB(t) \\
	dy(t) &=
		(\beta_p x w - (b + r_1) y)dt - \sigma_L y dB(t)\\
	dz(t) &=
		(b y - r_2 z)dt -\sigma_I z dB(t)\\
	dv(t) &=
		\left(-\beta_v v z  +(1-\theta-v)\frac{\mu}{N_v}\right)dt \\
	dw(t) &=
		\left(\beta_v v z + (\theta-w) \frac{\mu}{N_v}\right)dt
\end{align}























